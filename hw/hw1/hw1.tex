\documentclass{article}
\usepackage[margin=1.0in]{geometry}
\usepackage{authblk} %package for blocking authors, followed by blocking affiliation
\usepackage{url}
\usepackage{ulem} % when using ulem package, must change \emph to \it for italics.
\usepackage{hyperref}
\usepackage{array}
\usepackage{amssymb,amsmath,tabu}
\usepackage{hyperref}
\usepackage[super]{nth}
\setlength\parindent{0pt}
\usepackage[english]{babel}


\begin{document}
\title{Homework 1\\ SOCI 502B: Graduate Statistics (Fall 2017) \\ {\large{10 points}} \\ {\large{Due: December 13, 2017}}}
\author[*]{}
\date{}
\maketitle



\section*{Overview:}
In this assignment, you'll be completing 1 problem using SPSS (or R). This problem covers topics from Chapter 18 in your textbook.

\subsection*{Problem 1 (\textit{Chi-Square Test of Independence})}
\begin{itemize}
\item Using the GSS 2014 data set (\texttt{GSS2014\_final})\footnote{For SPSS, use \texttt{GSS2014\_final.sav}. For R, use the R script and \texttt{GSS2014\_final.csv}}, report a bivariate table for the relationship between the ``race'' variable (\texttt{race}) and the ``beliefs about racial inequality'' variable (\texttt{racdif4}).
\item First, recode the variables as such:
\begin{itemize}
\item race: 1 = white; 2 = black; 3 = other
\item racial inequality: 1 = racial inequality because blacks lack motivation; 2 = racial inequality \textbf{NOT} because blacks lack motivation\footnote{The survey question reads: ``On the average (Negroes/Blacks/African-Americans) have worse jobs, income, and housing than white people. Do you think these differences are... because most (Negroes/Blacks/African-Americans) just don't have the motivation or will power to pull themselves up out of poverty?''}
\end{itemize}
\item Include a clustered bar chart demonstrating the bivariate relationship.
\item Given the number of rows and columns in your table, report the degrees of freedom for the bivariate relationship between the ``race'' variable (\texttt{race}) and the ``beliefs about racial inequality'' variable (\texttt{racdif4}). 
\item What is the value of the Chi-Square statistic? 
\item Correctly and fully report and interpret the Chi-Square statistic (using the example from our lecture slides, including the $X^2$ value, the degrees of freedom, and the p-value)
\end{itemize}

\end{document}











