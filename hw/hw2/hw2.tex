\documentclass{article}
\usepackage[margin=1.0in]{geometry}
\usepackage{authblk} %package for blocking authors, followed by blocking affiliation
\usepackage{url}
\usepackage{ulem} % when using ulem package, must change \emph to \it for italics.
\usepackage{hyperref}
\usepackage{array}
\usepackage{amssymb,amsmath,tabu}
\usepackage{hyperref}
\usepackage[super]{nth}
\setlength\parindent{0pt}
\usepackage[english]{babel}


\begin{document}
\title{Homework 2\\ SOCI 502B: Graduate Statistics (Fall 2017) \\ {\large{10 points}} \\ {\large{Due: December 13, 2017}}}
\author[*]{}
\date{}
\maketitle



\section*{Overview:}
In this assignment, you'll be completing 1 problem using SPSS (or R). This problem covers topics from Chapter 9 in your textbook.

\subsection*{Problem 1 (\textit{$t$-test})}
\begin{itemize}
\item Using the GSS 2014 data set (\texttt{GSS2014\_final})\footnote{For SPSS, use \texttt{GSS2014\_final.sav}. For R, use the R script and \texttt{GSS2014\_final.csv}}, report the mean ``political (conservative) views'' (\texttt{polviews}) for each racial category (\texttt{race}).
\begin{itemize}
\item First, the race variable is listed as such:
\begin{itemize}
\item race: 1 = white; 2 = black; 3 = other
\end{itemize}
\item Recode race variable (into the same variable) as such:
\begin{itemize}
\item race: 1 = white; 2 = non-white
\end{itemize}
\item Next, the political views variable is listed as nominal. Change the variable from nominal to interval/ratio (scale). High scores indicate more conservativeness. 
\end{itemize}
\item Given the number of people in the sample, report the degrees of freedom for the bivariate relationship between the ``race'' variable (\texttt{race}) and the ``political (conservative) views'' variable (\texttt{polviews}). 
\item What is the value of the $t$ statistic? 
\item Correctly and fully report and interpret the $t$ statistic (using the example from our lecture slides, including the $t$ value, the degrees of freedom, and the p-value)
\end{itemize}

\end{document}











